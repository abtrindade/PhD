\thispagestyle{plain}
\begin{center}
%    \Large
%    \textbf{Automated Verification Applied to Stand-alone Solar Photovoltaic Systems}
%    
%    \vspace{0.4cm}
%    \large
%    Optimal Sizing and Project Validation
%    
%    \vspace{0.4cm}
%    \textbf{Alessandro Trindade}
%    
    \vspace{0.9cm}
    \textbf{Resumo}
\end{center}
Com custos decrescentes e com melhoria de desempenho, a implantação de sistemas de energia renovável está crescendo cada vez mais rapidamente no mundo. Em 2017, pela primeira vez, o número de pessoas sem acesso a eletricidade ficou abaixo de 1 bilhão, mas os dados quanto à universalização do acesso a energia ficaram aquém das metas globais. Particular atenção é dada aos sistemas isolados solares fotovoltaicos em áreas rurais ou onde a extensão da rede é inviável. Ferramentas para avaliar ou dimensionar projetos de eletrificação estão disponíveis, mas elas são baseadas em simulações que não cobrem todos os aspectos do espaço de projeto. Por outro lado, o uso de métodos formais para modelar e validar qualquer tipo de sistema está crescendo com o tempo, principalmente para encontrar bugs em sistemas complexos de \textit{hardware} e \textit{software}: seu objetivo é estabelecer a corretude do sistema com rigor matemático. O uso de métodos formais em sistemas elétricos é um assunto recente, com pesquisas sendo publicadas apenas nos últimos quatro anos. Além disso, a síntese automatizada nunca foi usada antes para obter um ótimo dimensionamento de sistemas solares fotovoltaicos . Esta tese marca duas conquistas principais: (1) a primeira aplicação de verificação de modelos de \textit{software} para verificar o projeto de um sistema isolado solar fotovoltaico, incluindo painel solar, controlador de carga, bateria, inversor e carga elétrica; e (2) uma abordagem confiável e automatizada para obter o dimensionamento ótimo de sistemas fotovoltaicos usando a síntese de programas onde cada componente e função de um sistema solar fotovoltaico é descrito, incluindo suas propriedades, e o modelo comportamental que representa o dimensionamento ótimo é sintetizado automaticamente. Relacionado à verificação formal, estudos de caso de sistemas fotovoltaicos reais instalados em cinco localidades diferentes são usados para avaliar a abordagem proposta e para compará-la com ferramenta de simulação especializada. Diferentes ferramentas de verificação são avaliadas também, a fim de comparar o desempenho e a confiabilidade dos resultados. Dados de aplicações práticas mostram a eficácia da abordagem proposta, onde condições específicas que levam a falhas em um sistema solar fotovoltaico são detalhadas apenas pelo método de verificação automatizado. Além disso, em relação ao uso da síntese de programas, propõe-se uma variante do método de síntese indutiva guiada por contraexemplos (CEGIS), com duas fases bem definidas: primeiro, ele sintetiza o dimensionamento de sistemas fotovoltaicos baseados em confiabilidade de energia, mas que pode não alcançar o menor custo; segundo, a solução proposta é então verificada iterativamente com um limite inferior via verificação de modelo simbólico. Se a etapa de verificação não falhar, o limite inferior será ajustado; e se falhar, o contraexemplo é fornecido com o dimensionamento ótimo, vinculando assim a resposta técnica da primeira fase à análise de custo da segunda fase. Os dados de equipamentos comerciais de diferentes fabricantes são fornecidos ao mecanismo de síntese e as soluções candidatas são derivadas da análise financeira do dimensionamento obtido. O método de síntese proposto é novo e sem precedentes para simplificar o projeto de sistemas fotovoltaicos. Resultados experimentais usando sete estudos de caso mostram que o nosso método de síntese é capaz de produzir em um tempo de execução aceitável o dimensionamento ótimo do sistema fotovoltaico, e um comparativo com uma ferramenta de simulação especializada e sistemas fotovoltaicos reais mostra a eficácia da abordagem adotada.
%Com custos decrescentes e com melhoria de desempenho, a implantação de sistemas de energia renovável está crescendo cada vez mais rapidamente no mundo. Em 2017, pela primeira vez, o número de pessoas sem acesso a eletricidade ficou abaixo de 1 bilhão, mas os dados quanto à universalização do acesso a energia ficaram aquém das metas globais. Particular atenção é dada aos sistemas isolados solares fotovoltaicos em áreas rurais ou onde a extensão da rede é inviável. Ferramentas para avaliar ou dimensionar projetos de eletrificação estão disponíveis, mas elas são baseadas em simulações que não cobrem todos os aspectos do espaço de projeto. Por outro lado, o uso de métodos formais para modelar e validar qualquer tipo de sistema está crescendo com o tempo, principalmente para encontrar bugs em sistemas complexos de \textit{hardware} e \textit{software}: seu objetivo é estabelecer a corretude do sistema com rigor matemático. O uso de métodos formais em sistemas elétricos é um assunto recente, com pesquisas sendo publicadas apenas nos últimos quatro anos. Além disso, a síntese automatizada nunca foi usada antes para obter um ótimo dimensionamento de sistemas solares fotovoltaicos nunca foi feito antes. Esta tese marca duas conquistas principais: (1) a primeira aplicação de verificação de modelos de software para verificar o projeto de um sistema isolado solar fotovoltaico, incluindo painel solar, controlador de carga, bateria, inversor e carga elétrica; e (2) uma abordagem confiável e automatizada para obter o dimensionamento ótimo de sistemas fotovoltaicos usando a síntese de programas onde cada componente e função de um sistema solar fotovoltaico é descrito, incluindo suas propriedades, e o modelo comportamental que representa o dimensionamento ótimo é sintetizado automaticamente. Relacionado à verificação formal, estudos de caso de sistemas fotovoltaicos reais instalados em cinco localidades diferentes são usados para avaliar a abordagem proposta e para compará-la com ferramenta de simulação especializada. Diferentes ferramentas de verificação são avaliadas também, a fim de comparar o desempenho e a confiabilidade dos resultados. Dados de aplicações práticas mostram a eficácia da abordagem proposta, onde condições específicas que levam a falhas em um sistema solar fotovoltaico são detalhadas apenas pelo método de verificação automatizado. Além disso, em relação ao uso da síntese de programas, propõe-se uma variante do método de síntese indutiva guiada por contraexemplos (CEGIS), com duas fases bem definidas: primeiro, ele sintetiza o dimensionamento de sistemas fotovoltaicos baseados em confiabilidade de energia, mas que pode não alcançar o menor custo; segundo, a solução proposta é então verificada iterativamente com um limite inferior via verificação de modelo simbólico. Se a etapa de verificação não falhar, o limite inferior será ajustado; e se falhar, o contraexemplo é fornecido com o dimensionamento ótimo, vinculando assim a resposta técnica da primeira fase à análise de custo da segunda fase. Os dados de equipamentos comerciais de diferentes fabricantes são fornecidos ao mecanismo de síntese e as soluções candidatas são derivadas da análise financeira do dimensionamento obtido. O método de síntese proposto é novo e sem precedentes para simplificar o projeto de sistemas fotovoltaicos. Resultados experimentais usando sete estudos de caso mostram que o nosso método de síntese é capaz de produzir em um tempo de execução aceitável o dimensionamento ótimo do sistema fotovoltaico, e um comparativo com uma ferramenta de simulação especializada e sistemas fotovoltaicos reais mostra a eficácia da abordagem adotada.

\textit{Palavras-chave}: Verificação formal; verificação automatizada; verificação de modelos; síntese de programa; sistemas elétricos; sistema solar fotovoltaico