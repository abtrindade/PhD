%\section{Intro to Automated Formal Synthesis Optimization}
This section presents the case studies used to evaluate our proposed approach. 
We also compare our approach with a simulation tool, named HOME Pro. 
The version and command-line of each verifier adopted, 
the computing setup, the objectives of the experimental phase, 
and the results itself are also described.

%---------------------------------------------------------------------------
\section{Optimization Case studies} 
%---------------------------------------------------------------------------

We have performed seven stand-alone PV system case studies to evaluate 
our proposed synthesis approach, as described in the first column of 
Table~\ref{tab1} (named Specification). These case studies were defined 
based on usual electrical load found in riverside communities of the 
Amazon State in Brazil~\cite{abs-1811-09438, Agrener2013}. 
%
\begin{table}
\caption{Case studies and results: optimization of stand-alone PV systems.}\label{tab1}
\begin{scriptsize}
\begin{tabular}{|c|c|c|c|c|}
\hline
\hline
Tools & \makecell{CBMC 5.11 \\(MiniSat 2.2.1)}& \makecell{ESBMC 6.0.0 \\(Boolector 3.0.1 /\\Z3 4.7.1)}& \makecell{CPAchecker 1.8\\(MathSAT 5.5.3)}& HOMER Pro 3.13.1\\
\hline
\hline
Specification & Result & Result & Result & Result \\
\hline
\makecell{\textbf{Case Study 1}\\Peak:342W\\Surge:342W \\E:3,900Wh/day\\Autonomy:48h} & OM & TO / IF & \makecell{SAT (172.03 min) \\NTP:1$\times$340W (1S)\\NBT:8$\times$105Ah (2S-4P)\\Controller 15A/75V\\Inverter 700W/48V\\LCC: US\$ 7,790.53} & \makecell{(Time: 0.33 min)\\2.53 kW of PV\\NBT:12$\times$83.4Ah (2S-6P)\\0.351kW inverter\\LCC: US\$ 7,808.04}\\
\hline
\makecell{\textbf{Case Study 2}\\Peak:814W\\Surge:980W\\E:4,880Wh/day\\Autonomy:48h} & OM & TO / IF & \makecell {SAT (228.7 min) \\NTP:2$\times$330W (2S)\\NBT:10$\times$105Ah (2S-5P)\\Controller 20A/100V DC\\Inverter 1,200W/24V \\LCC: US\$ 8,335.90} & \makecell{(Time: 0.18 min)\\3.71 kW of PV\\NBT:20$\times$83.4Ah (2S-10P)\\0.817kW inverter\\LCC: US\$ 12,861.75} \\
\hline
\makecell{\textbf{Case Study 3}\\Peak:815W\\Surge:980W\\E:4,880Wh/day\\Autonomy:12h} & OM & TO / IF & \makecell {SAT (166.13 min) \\NTP:4$\times$150W (4S)\\NBT:4$\times$80Ah (2S-2P)\\Controller 15A/100V DC\\Inverter 1,200W/24V \\LCC: US\$ 7,306.27} & Not possible \\
\hline
\makecell{\textbf{Case Study 4}\\Peak:253W\\Surge:722W\\E:3,600Wh/day\\Autonomy:48h} & OM & TO / IF & \makecell {SAT (143.71 min) \\NTP:4$\times$150W (4S)\\NBT:10$\times$80Ah (2S-5P)\\Controller 15A/75V\\Inverter 750W/24V \\LCC: US\$ 7,816.31} & \makecell{(Time: 0.23 min)\\2.42 kW of PV\\NBT:12$\times$83.4Ah (2S-6P)\\0.254kW inverter\\LCC: US\$ 7,677.95}\\
\hline
\makecell{\textbf{Case Study 5}\\Peak:263W\\Surge:732W\\E:2,500Wh/day\\Autonomy:48h} & OM & TO / IF & \makecell {SAT (134.93 min) \\NTP:1$\times$340W (1S)\\NBT:6$\times$105Ah (2S-3P)\\Controller 15A/75V\\Inverter 400W/24V \\LCC: US\$ 7,252.14} & \makecell{(Time: 0.18 min)\\1.59 kW of PV\\NBT:10$\times$83.4Ah (2S-5P)\\0.268kW inverter\\LCC: US\$ 6,175.57} \\
\hline
\makecell{\textbf{Case Study 6}\\Peak:322W\\Surge:896W\\E:4,300Wh/day\\Autonomy:48h} & OM & TO / IF & \makecell {SAT (235.75 min) \\NTP:2$\times$200W (2S)\\NBT:10$\times$105Ah (2S-5P)\\Controller 15A/75V\\Inverter 400W/24V \\LCC: US\$ 8,287.23} & \makecell{(Time: 0.22 min)\\3.15 kW of PV\\NBT:14$\times$83.4Ah (2S-7P)\\0.328kW inverter\\LCC: US\$ 9,112.45} \\
\hline
\makecell{\textbf{Case Study 7}\\Peak:1,586W\\Surge:2,900W\\E:14,000Wh/day\\Autonomy:48h} & OM & TO / IF & TO & \makecell{(Time: 0.20 min)\\12.5 kW of PV\\NBT:66$\times$83.4Ah (2S-33P)\\1.60kW inverter\\LCC: US\$ 41,878.11} \\
\hline
\hline
\end{tabular}
\\Legend: OM = out of memory; TO = timeout; IF = internal failure, E = energy.
\end{scriptsize}
\end{table}


%---------------------------------------------------------------------------
\section{Tools} 
%---------------------------------------------------------------------------

Three start-of-art verification tools, CBMC\footnote{Command-line: \$ cbmc -\phantom{}-unwind 100 filename.c -\phantom{}-trace}, ESBMC\footnote{Command-line: \$ esbmc filename.c -\phantom{}-no-bounds-check -\phantom{}-no-pointer-check -\phantom{}-unwind 100 -\phantom{}-boolector}, %UAutomizer\footnote{Command-line: \$ ./Ultimate -tc config/AutomizerReach.xml -s config/svcomp-Reach-32bit-Automizer\_Default.epf -i filename.c -\phantom{}-traceabstraction.limit.analysis.time 900 -\phantom{}-traceabstraction.stop.after.first.violation.was.found false -\phantom{}-cacsl2boogietranslator.overapproximate.operations.on.floating.types false -\phantom{}- cacsl2boogietranslator.assume.nondeterminstic.values.are.in.range false -\phantom{}-rcfgbuilder.add.additional.assume.for.each.assert true -\phantom{}-rcfgbuilder.simplify.code.blocks true -\phantom{}-rcfgbuilder.size.of.a.code.block LoopFreeBlock}, 
and CPAchecker\footnote{Command-line: \$ scripts/cpa.sh -heap 64000m -config config/bmc-incremental.properties -spec config/specification/sv-comp-reachability.spc filename.c} were used as our verification engine to compare our approach effectiveness and efficiency. Note that ``incremental'' ESBMC with the SMT solver Z3 was tried\footnote{Command-line: \$ esbmc filename.c -\phantom{}-no-bounds-check -\phantom{}-no-pointer-check -\phantom{}-unwind 100 -\phantom{}-smt-during-symex -\phantom{}-smt-symex-guard -\phantom{}-z3} as an alternative to use less computing memory. The Simulation tool HOMER Pro version $3.13.1$ was used for comparative purpose.

%---------------------------------------------------------------------------
\section{Setup} 
%---------------------------------------------------------------------------

All experiments regarding the verification tools were conducted 
on an otherwise idle Intel Xeon CPU E5-4617 ($8$-cores) with 
$2.90$ GHz and $64$ GB of RAM, running Ubuntu $16.04$ LTS $64$-bits. 
Related to HOMER Pro, we have used an Intel Core i5-$4210$ ($4$-cores), 
with $1.7$ GHz and $4$ GB of RAM, running Windows 10. 
Our experiments were performed with a predefined timeout of $240$ minutes.

%---------------------------------------------------------------------------
\section{Objectives} 
%---------------------------------------------------------------------------

Our evaluation aims to answer two experimental questions: 

\begin{enumerate}

\item[EQ1] \textbf{(soundness)} does our automated synthesis approach provide correct results?

\item[EQ2] \textbf{(performance)} how do the software verifiers compare to each other?

\end{enumerate}

%---------------------------------------------------------------------------
\
section{Results}  
%---------------------------------------------------------------------------

CPAchecker was able to synthesize the optimal sizing in six 
out of seven case studies: the result was produced within 
the time limit, which varied from $134.71$ to $235.75$ minutes. 
Only case study $7$ led to a \textit{timeout} result, i.e., 
it was not solved within $240$ minutes. However, if we remove 
this timeout limitation from CPAchecker, the verifier is 
able to solve the optimization in $44.97$ hours. 
The violation (SAT result) indicated in Table~\ref{tab1} 
is the $assert$ of line $22$ from Algorithm~\ref{alg:verification-algorithm}. %The results were tested by manual PV sizing and were sound (\textit{RQ1}). %, linking a feasible technical solution with the lowest cost possible, considering the equipment that were inputted to the code. 

CBMC and ESBMC are unable to produce any conclusive result. 
Situations of \textit{internal failure}, \textit{timeout}, 
or \textit{out of memory} occurred; this partially answers 
the \textit{EQ2}. Note that the internal failure presented 
by ESBMC was a Z3 solver issue (a bug); this will demand 
an updated version of ESBMC to fix this issue. Similarly to CPAchecker, 
if we remove the timeout from ESBMC with the SMT solver Boolector, 
then the verifier is able to obtain the automated synthesis 
in $73.18$ hours for the case study $2$. CBMC, in the other hand, only could present some result if the RAM memory of the system was bigger to avoid the memory out issue.

Related to HOMER Pro, it was able to evaluate six case studies, 
and within a time shorter than $30$ seconds, which was much 
faster than our automated synthesis tool (cf.~\textit{EQ2}). 
Case study $3$ was not possible to be simulated since HOMER Pro 
does not have the feature of adjusting the battery autonomy, i.e., 
the tool always tries to feed with electricity the given load 
during $365$ days/year. We have also noted other HOMER Pro drawbacks:

\begin{itemize}
\item There exists no explicit charge controller 
as a system equipment. HOMER Pro includes automatically 
a controller just to simulate the charge/discharge 
of batteries and to meet the load requirement; however, 
without costs or even with electrical characteristics 
as maximum current and voltage, which are common during PV sizing;
\item HOME Pro demands to include some battery specification 
to initiate the optimization; however, it does not change 
the electrical specifications during the simulation; 
the presented results are multiples of the original 
battery type suggested by the user. As example, it was 
started with a $83.4$ Ah lead-acid battery and during 
the simulation, HOMER Pro did not try to use other capacities or types;
\item HOMER Pro does not present the optimal solution 
in terms of connections of arrays of PV panels, just the 
total in terms of power, i.e., it does not present neither models 
and the power of each PV panel nor the total of panels in series or parallel. 
\end{itemize}

%%%%%%%%%%%%%%%%%%%%%%%%%%%%%%%%%%%%%%%%%%%%%%%%%%%%%%%%%%%%%%%%%%%%
\section{Comparison between Formal Synthesis and HOME Pro}
%%%%%%%%%%%%%%%%%%%%%%%%%%%%%%%%%%%%%%%%%%%%%%%%%%%%%%%%%%%%%%%%%%%%

Comparing the results between the formal synthesis with CPAchecker 
and HOMER Pro, we observed that most results are quite similar, 
in terms of technical solution and cost (cf. Table~\ref{tab1}). 

Particularly related to LCC, the cost was very close in cases 
$1$, $4$, $5$ and $6$, with difference varying from $0.23$\% to $17.4$\%. 
Even adopting the same price per kW to the PV panels, 
inverters, and batteries, HOMER Pro does not use costs 
related to charge controllers, which were introduced into the 
CPAchecker modeling. The premise used in CPAchecker to adopt 
a fixed annual cost for operation and maintenance can produce 
some impact as well at this discrepancy; however, it is not significant
since this annual cost is too small when compared to the resulting LCC value.

However, there exists a huge divergence in case study $2$, 
where the costs presented by HOMER Pro were $54$\% higher 
than our automated synthesis tool, probably because the 
operation and maintenance costs assumed by our automated 
synthesis tool were underestimated to that specific load. 

In general, the size of the PV panels and battery bank were 
bigger in HOMER Pro than with our formal synthesis approach, 
and that discrepancy is not easy to address without some real 
systems validation. The mathematical models are different and 
particular parameters can be tuned as well in each approach, 
and that can justify the difference, which was presented in all 
the case studies. As comparative, let's consider case study $1$: 
the optimal solution provided by HOMER Pro demands $7$ $\times$ 
more PV panels than the solution presented by our synthesis tool, 
and HOME Pro does not show the arrangement of arrays 
(i.e., the number of series and parallel PV panels); 
the battery bank presented by HOMER Pro provides $500.4$ Ah 
of capacity ($6 \times 83.4$), while our synthesis tool 
presented an optimal solution with $420$ Ah of total capacity 
($4 \times 105$). 

Just to compare the results obtained from the optimization 
with the real-world, the authors had four PV systems deployed 
and monitored since June $2018$ in a riverside community 
in the Amazonas State in Brazil, with similar demands 
presented by case studies $1$, $4$, $5$, and $6$, 
always with a $3$ $\times$ $325$ W ($3$S) panels and 
$4$ $\times$ $220$ Ah ($2$S-$2$P $= 440$ Ah) 
lead-acid batteries. These solutions are more close 
to the result presented by our formal synthesis 
approach than HOMER Pro, thereby showing that our 
solution is sound, which answers \textit{EQ1}.

Related to the inverters, HOMER Pro suggests a value in 
kW very close to the peak of every case study, and it 
is just a reference value and not a commercial value of 
the employed inverter. Our synthesis tool, however, 
presents inverters that are commercial and can be found 
off-the-shelf. Therefore is a PRO to the formal synthesis method.

Concerning to charge controllers, as we reported in 
the previous section, HOMER Pro does not include it 
as an explicit equipment in its mathematical model, 
only our synthesis tool presents a commercial controller 
and includes it during the cost analysis. Therefore, 
the formal synthesis method presents more reliable results than
HOME Pro.

Case study $7$ was not solved by our synthesis tool 
within the time limit established during the experimental 
phase. Case study $3$ was not possible to simulate in HOMER Pro, 
because its restriction does not allow one to set the battery autonomy, 
thus resting both without parameters to comparative.

Summarizing, our synthesis tool is capable to present a 
solution, which is far detailed and close to the commercial 
reality than the solution presented by HOMER Pro. 
In particular, our automated synthesis method 
can provide all the details of every component of 
a PV system solution, with complete electrical details 
from data sheet of manufacturers, including 
the model of the component, nominal current and voltage. 
In this respect, even the name of the manufacturer 
can be presented (in Table~\ref{tab1} it was removed 
to avoid some advertising).
%used with the SMT incremental mode\footnote{Command-line: \$ esbmc filename.c -\phantom{}-no-bounds-check -\phantom{}-no-pointer-check -\phantom{}-unwind 100 -\phantom{}-smt-during-symex -\phantom{}-smt-symex-guard -\phantom{}-z3} enabled with the goal of reducing memory usage; we have also used the SMT solver Z3 version 4.7.1~\cite{DeMoura}.

%%%%%%%%%%%%%%%%%%%%%%%%%%%%%%%%%%%%%%%%%%
\section{Threats to validity}
%%%%%%%%%%%%%%%%%%%%%%%%%%%%%%%%%%%%%%%%%%

We have reported a favorable assessment of our formal synthesis method. 
Nevertheless, we have also identified three threats to the validity 
of our experimental results, which can be further assessed and 
constitute future work: ($1$) improvement of the power reliability 
analysis: to include loss of load probability or loss of power 
supply probability, which can make the analysis more accurate; 
($2$) the cost analysis is well tailored to the Amazon region of Brazil; 
however, a broad analysis from other isolated areas must be 
performed in order to make the optimization general in terms 
of applicability; ($3$) to deploy at the field some PV systems 
sized using our synthesized results in order to validate it.

\section{Conclusion}