Here we describe the two tools created during the PhD process to implement and validate the two scientific methodologies of the thesis.

\section{Tools}

During the PhD process, the techniques from computer science and proposed to tackle issues related to solar photovoltaic systems became algorithms and codes. Those codes became tools, and the tools were used during comparative with simulation tool in order to validate the technique.

In general, we pursuit to write code that could be verification tool independent, i.e., that can be used in any model checking tool without adaptation.

We wrote the tools in Language C, without the use of '\# include', or '\# define'. Based on the fact that there is no use of compiler library, it was necessary to write some functions which perform specific calculations according to the adopted mathematical models (logarithm of natural order and exponential).

\subsubsection{Automated Verification Tool (PV System validation)}

The code can be divided in four blocks:

\begin{itemize}
\item \textbf{Block 1}: global variables declaration, with weather data from location (minimum and maximum solar irradiance and temperature, for 24-hour of the day) as array, requirements, and data sheet information from each equipment of the PV system that will be verified. 

\item \textbf{Block 2}: support functions (logarithm of natural order, exponential)

\item \textbf{Block 3}: PV system specific functions (charge battery, discharge battery, PV panel generation, sizing check)

\item \textbf{Block 4}: main code with charge, discharge battery control, based on PV generation from solar panels and the state of charge form batteries, having the commitment to deliver the power and energy that the house demands, according with the restrictions form the sized system.
\end{itemize}

\subsubsection{Automated Synthesis Tool (PV System Optimal Sizing)}

The code can be divided in three blocks:

\begin{itemize}
\item \textbf{Block 1}: global variables declaration, with weather data from location, requirements, and data sheet information from each equipment of the PV system that will be verified. 

\item \textbf{Block 2}: support functions (start value with lowest cost for the list of equipment, synthesis phase function to obtain feasible technical solutions)

\item \textbf{Block 3}: main code with the iterative control of the verify phase
\end{itemize}

\section{How to use it}
The only block that demands user intervention from user in order to input data is the block 1 of both tools.

Regarding the automated verification tool, it is necessary to input the following data (editing and modifying the code):

\begin{itemize}
\item Minimum and maximum solar irradiance (average value for the 24-hour of the day) from location
\item Minimum and maximum temperature (average for every month of the year)
\item Local insolation (average number of sun hours by day at the location)
\item Number of solar panels in series
\item Number of solar panels in parallel
\item Number of series-connected cells from each solar panel
\item Nominal Operating Cell Temperature
\item Reference solar irradiance
\item Reference temperature
\item Short-circuit current temperature coefficient
\item Open-circuit voltage temperature coefficient
\item Reference short-circuit current
\item Reference open-circuit voltage
\item Reference maximum current, voltage, and power
\item Minimum MPPT voltage
\item Operation current
\item Solar panel efficiency
\item Panel area in square meters
\item DC-bus voltage
\item Individual battery voltage
\item Battery bank capacity
\item State of charge limit
\item Number of batteries in series
\item Number of batteries in parallel
\item Autonomy of the batteries
\item Efficiency from batteries
\item Float, absorption and bulk battery voltages
\item Charge controller efficiency
\item Charge controller current
\item Charge controller maximum voltage
\item Inverter output voltage
\item Inverter AC reference power
\item Inverter maximum (surge) AC power reference
\item AC voltage (outlet standard)
\item Power demand from the house
\item Surge demand from the house
\item Energy consumption from the house
\end{itemize}

And regarding the automated synthesis tool, it is necessary the following data (editing and modifying the code):

\begin{itemize}
\item Minimum and maximum solar irradiance (average value for the 24-hour of the day) from location
\item Minimum and maximum temperature (average for every month of the year)
\item Local insolation (average number of sun hours by day at the location)
\item DC-bus voltage
\item Autonomy of the batteries
\item State of charge limit
\item Individual battery voltage
\item Reference solar irradiance
\item Reference temperature
\item For every item of solar PV panel list, as a matrix (area, efficiency, number of series-connected cells, Nominal Operating Cell Temperature, short-circuit current temperature coefficient, open-circuit voltage temperature coefficient, reference short-circuit current, reference open-circuit voltage, reference maximum power, reference maximum current, reference maximum voltage, maximum voltage on NOCT, cost in US dollars)
\item For every item of batteries list, as a matrix (efficiency, voltage, capacity $C_{20}$), bulk voltage, float voltage, cost in US dollars)
\item For every item of charge controller list, as a matrix (efficiency, nominal current, voltage  output, minimum MPPT voltage, maximum current, cost in US dollars)
\item For every item of inverter list, as a matrix (efficiency, input DC voltage, output AC voltage, reference AC power, reference maximum AC power, cost in US dollars)
\item AC voltage (outlet standard)
\item Power demand from the house
\item Surge demand from the house
\item Energy consumption from the house
\end{itemize}

\section{Automated Verification with Incremental ESBMC}
Here 123

\section{Automated Synthesis with CPAchecker}
Here 4321


