Here we present the conclusions about the research done and the results obtained during the Doctoral process, including the two scientific methods or techniques from computer science that were used to tackle solar PV systems issues, as validation over time and sizing optimization. Moreover, a list of future contributions, which probably will foster the creation of a research group that applies model checking to electrical systems (not only renewable sources), concludes the work.

%----------------------------------------------
\section{Main Contributions}
%----------------------------------------------

Our contributions can be split in two, each one concerning the scientific methods proposed with the use of model checking to solar PV systems, and the tools created to validate it.

Concerning the first contribution presented in ~\autoref{chap:automatedverification}, we described and evaluated an automated verification method to validate a given stand-alone PV system design and check whether it meets its user and electrical requirements using software model checking techniques.  There exist various possibilities to perform that validation, such as testing, laboratory measuring, or even simulation (used before field deployment, as our proposal). \textcolor{red}{this sentence is unclear....} However, we showed, supported by an algorithm that uses our approach, moreover supported by case studies, by monitoring data from field, and from an interview with the owners, that our approach is feasible and effective.

It was considered five case studies from real PV systems deployed in five different sites, ranging from $700$ W to $1,200$ W (inverter specification), and three state-of-art verification engines were considered (ESBMC, CBMC, and CPAchecker). Although the verification method that we proposed takes longer than simulation methods, it can avoid the battery bank oversize produced by the simulation tool in houses $1$, $2$, $3$, and $4$. This result was validated by the sized systems deployed in the field.

Moreover, it was not necessary to ``adapt'' the tool, since the specialized simulation tool (HOMER Pro) had to be adjusted in order not to perform the optimization of the system. \textcolor{red}{this sentence is unclear...} Besides, only the proposed method of automated verification was possible to validate a system with a specific battery autonomy, since HOMER Pro does not have this feature.

%In a nutshell, our proposed PV system validation method, who check the size of inputted PV system, and verifies it over the time, was feasible and effective, supported by a algorithm written to implement the method, by a comparative with a commercial simulation tool, and by real data collect at the field.
%
\textcolor{red}{this sentence needs to be rewritten... when you use better, less, etc, you need to use ``than'' and add the second sentence for the comparison.}
Related to the verification engines comparative, the ESBMC with the Z3 solver executed in the 'incremental SMT' configuration presented the better performance (around four times faster than CBMC), used less RAM memory (less than $2$ GB when compared to $9.2$ GB of CBMC and 19.2GB of CPAchecker), and all the results were sound because the PV owners and the monitoring system validated the possible flaws that the system could be presented in the field.
%New tests must be performed, in an improved setup (i.e., better computing performance) with the goal of obtaining verification results in shorter times than the ones obtained in this study. 

Furthermore, we have also contributions from the method described in ~\autoref{chap:automatedsynthesis}. We have described and evaluated an automated synthesis method to obtain the optimal size of the PV system using software model checking techniques. The focus was on the synthesis method to obtain the optimal solution based on formal methods, which can cover better the design-space as opposed to simulation tools. Our thesis produced a methodological research with innovative value regarding the first use of automated synthesis for optimal sizing of solar PV systems.

\textcolor{red}{These sentences are too long and difficult to understand.}
We proposed a variant of CEGIS synthesis process, working with just one feasible solution from the {\sc Synthesize} phase, and a refinement from the iterative search from the {\sc Verify} phase, which allows the optimization of stand-alone PV systems with the best compromise between power reliability and system cost analysis. Our algorithm that implements this method uses a database of commercial equipment from the marketing, including the price. In order to validate our proposed method, we considered seven case studies from PV systems in two different sites of the Amazonas State in Brazil, ranging from $253$\,W to $1,586$\,W peak; and three state-of-art verification engines were considered (ESBMC, CBMC, and CPAchecker), in addition to a specialized off-the-shelf simulation tool (HOMER Pro) to compare the results. In terms of performance and better results, CPAchecker was the winner and used for the comparative with the simulation tool. 

In summary, our synthesis proposal is capable of presenting a solution, which is far detailed and close to the commercial reality than the solution presented by HOMER Pro. In particular, our method can provide all the details of every component of a PV system solution, with complete electrical details from datasheet of manufacturers, including the component model, nominal current, and voltage. We also cover the charge controller, which is unavailable in HOMER Pro. Note that our automated synthesis tool took longer to find the optimal solution than HOMER Pro; however, the presented solution is sound and complete; it also provides a list of equipment to be bought from manufacturers. %Moreover, we extended the CEGIS synthesis method and implemented this extension within our proposed formal synthesis tool, which allows the optimization of stand-alone PV systems with the best compromise between power reliability and system cost analysis.


%----------------------------------------------
\section{Future Work Directions}
%----------------------------------------------

As the focus here was stand-alone solar PV systems, one future work will also consider other types of renewable energy and even hybrid ones to allow the method to verify and to obtain the optimal sizing of typical rural electrification systems.

Related to automated sizing, the author plans to improve the power reliability analysis, to address the restriction to only allow automated synthesis of riverside communities in the Amazonas state (Brazil), and to validate some cases with deployed PV systems in isolated communities.

\textcolor{red}{This sentence is too long and difficult to understand.}
It is planned to develop the code of a general-purpose simulation tool, using the same mathematical model employed for the model checking methods of this work, to include it in the comparative in order to cover all the possibilities to validate or optimize solar PV systems.

One future long term direction is to foster the creation of a research group in automated verification and synthesis applied to electrical systems.

%----------------------------------------------
\section{Concluding Remarks}
%----------------------------------------------

\textcolor{red}{This sentence is too long and difficult to understand.}
The automated verification and automates synthesis methods demonstrate the effectiveness and the potential of use in stand-alone solar PV systems. Some issues must be addressed in order to improve the tools, as performance and the human-computer interaction. However, the work has potential, and it is promising \textcolor{red}{how is it promising?}.

\textcolor{red}{This sentence is too long and difficult to understand.}
About license and the use of tools that the author employed at this work, must be noticed that HOMER Pro is available only for Microsoft Windows and its annual standard subscription costs US\$ $504.00$~\cite{HOMER}; the verifiers, as CPAchecker, ESBMC and CBMC, and their solvers, they are based on open-source software license and usually allows the users to use, modify, and distribute licensed product freely. Permission is granted, free of charge, to use this software for evaluation and research purposes (and that is an advantage when compared to commercial tools); therefore, some of the licenses do not allow the software to be used in a commercial context.

Based on the fact that the tools detailed here have not similarity in past work in the world, can be concluded that it was produced original research that expands the boundaries of knowledge, putting cutting-edge computer science methodologies to solve typical electrical engineering problems and to improve the design of systems.

\textcolor{red}{this sentence is tool informal with the use of ``you''.}
And last, but not least significant to report, besides the fact that the goal of a PhD is to contribute to the body of human research, one of the great things about PhD is you will be able to conduct your research. In that direction, we can say that with the technique proposed here (formal methods using automated verification and model checking) to energy sector, there exists a vast field of future applications and developments, using different renewable sources, as biomass, wind, or even hybrid, with different mathematical models and requirements; and including smart grids, or electrical systems in general. Few research has been done until now, and the scenario is auspicious from now on.

