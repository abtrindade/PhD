introduction text here kadhew nrjrenasmnfs,mgfs gfsdf gdsf dfsg dsgh bv dzfg df gdf gdf gdf gdf gd fg dfg dfg

Among the options of renewable sources, there are hydro, wind, and solar photovoltaic (PV). Only a niche market a few years ago, PV are now becoming a mainstream electricity provider, changing the way the world is powered. Based in data from \cite{EPIA}, there was an increase of approximately 30\% from 2013 to 2014 in terms of new installations of PV. The use of the power generation technology with renewable energy source is developing rapidly due to the industrial development \cite{Yatimi}. The renewable energy leads to an advance all over the world by protecting the environment: it is clean (low greenhouse gases emissions), operating silently, long lifetime, low maintenance, absence of fuel cost and inexhaustible \cite{Noroozian}. Renewable energy, and particularly the power generation from solar energy using photovoltaic (PV) panels, has emerged as an alternative to fossil or nuclear fuel generation. The increase in a number of PV systems installed all over the world brings the need for proper modeling and simulation tools for researcher and practitioners involved in their application. 

In order to validate or even simulate a potential PV solution, there are a myriad of tools. However, to the best of our knowledge, this work can be the first work to perform automated verification of a solar PV off-grid project solution.  

According to \cite{SEIA} and \cite{Chauhan}, solar is the most abundant source of renewable energy on earth. In SPV system, solar radiations are captured from the sun and turned into electricity using SPV cells made of silicon and other materials. SPV systems can be classified into grid-connected and stand-alone systems. At this proposal, only off-grid or stand-alone systems will be considered. The utilization of solar energy in off-grid mode has the potential to meet the energy need for remote rural areas of developing countries.  

The PV cell in a solar photovoltaic system, as defined in \cite{Rawat}, is a semiconductor device, which directly converts the solar radiation into electrical energy. Apart from PV modules; the PV systems consist of battery bank, controller, and inverter, plus the load. Moreover, the optimum sizing of these devices is important for reliable operation. Therefore, these systems need to be designed properly according to the site, land area available, load requirement, load pattern, environmental conditions and economics in order to utilize available resources efficiently and economically \cite{Rawat}.

\subsection{Problem's Definition}
 
In order to address different aspects of a SPV project, there is a myriad of public domain and commercial software available. According to \cite{Brooks}, the capabilities of these tools range from simple solar resource and energy production estimative, to site survey and system design tools, to complex financial analysis software (with optimization). Some tools also provide support to rebate programs applications and tax incentives (specific to each country or region), while other programs and worksheets focus on the technical aspects of system sizing and design.  

Manufacturers and integrators have yet their proprietary software to perform various system sizing, with the drawback of include just their own products among the possibilities of choice, what restrict the solution. 

Changing the subject to the design of complex systems, more time is spent on verification than on construction, as shown in \cite{Baier}. Formal methods based on model checking offer great potential to obtain a more effective and faster verification in the design process. Programs, and more generally computer systems, may be viewed as mathematical objects with behavior that is, in principle, well determined. This makes it possible to specify programs using mathematical logic, which constitutes the intended (correct) behavior. Then, one can try to give a formal proof or otherwise establish that the program meets its specification, as defined by \cite{Trindade}. 

This area of research is referred as formal methods as defined in \cite{Clarkeetal}. Their aim is to establish system correctness with mathematical rigor. 

In the recent decades, research in formal methods has led to the development of very promising verification techniques that facilitate the early detection of bugs in order to ensure the correctness of the system. 

Model-based verification techniques are based on models that describe the possible system behavior in a mathematically precise and unambiguous manner. Thus, such problems as incompleteness, ambiguities, and inconsistencies, which normally are discovered only in later stages of the design, can be detected in advance, as described by \cite{Trindade}. Algorithms can verify the system model through systematically explore all their states.

\subsection{Objectives}
Objectives here

\subsection{Contributions}
Contributions text subsection here
 
\subsection{Related Work}

\subsection{Thesis Organization}
Text here