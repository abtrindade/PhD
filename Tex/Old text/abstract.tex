\thispagestyle{plain}
\begin{center}
%    \Large
%    \textbf{Automated Verification Applied to Stand-alone Solar Photovoltaic Systems}
%    
%    \vspace{0.4cm}
%    \large
%    Optimal Sizing and Project Validation
%    
%    \vspace{0.4cm}
%    \textbf{Alessandro Trindade}
%    
    \vspace{0.9cm}
    \textbf{Abstract}
\end{center}
With declining costs and increasing performance, the deployment of renewable energy systems is now growing faster than in the previous decade. In 2017, for the first time, the number of people without access to electricity dipped below 1 billion, but trends on energy access likewise fall short of global goals. Particular attention is given to stand-alone solar photovoltaic systems in rural areas or where grid extension is unfeasible. Tools to evaluate or to size electrification projects are available, but they are based on simulations that do not cover all aspects of the design space. However, the use of formal methods to model and validate any system is growing with time, mainly to find bugs in sophisticated hardware and software systems: they aim to establish system correctness with mathematical rigor. The use of formal methods in electrical systems is a new subject, with research being published only in the last four years.
Moreover, the use of automated synthesis in order to obtain the optimal sizing of solar photovoltaic systems was never done before. This Ph.D. thesis marks two mains achievements. First, the application of software model checking to formally verify the design of a stand-alone solar photovoltaic system, including solar panel, charge controller, battery, inverter, and electric load. Second, a sound and automated approach to obtain optimal sizing of stand-alone photovoltaic systems using program synthesis. Related to formal verification, we used case studies from real photovoltaic systems deployed in five different sites, ranging from 700 W to 1,200 W, in order to evaluate the proposed approach and to compare that with specialized simulation tool. Different verification tools are evaluated as well in order to compare performance and soundness. Data from practical applications show the effectiveness of our proposed approach, where specific conditions that lead to failures in a photovoltaic solar system are only detailed by the automated verification method.
Moreover, concerning the use of program synthesis, we propose a variant of counterexample guided inductive synthesis (CEGIS) approach. This variant has two phases linking the technical and cost analysis. First, we synthesize a feasible candidate based on power reliability, but that may not achieve the lowest cost. Second, the candidate is then verified iteratively with a lower bound cost via symbolic model checking. If the verification step succeeds, the lower bound is adjusted; if it fails, a counterexample provides the optimal solution. The proposed synthesis method is novel and unprecedented to streamline the design of photovoltaic systems. Experimental results using seven case studies show that our synthesis method can produce within an acceptable run-time the optimal system sizing. We also present a comparative with a specialized simulation tool over real photovoltaic systems to show the effectiveness of our approach, which can provide a more detailed and accurate solution than that simulation tool.

\textit{Keywords}: Formal verification; automated verification; model checking; program synthesis; electrical systems; and solar photovoltaic systems.
