\thispagestyle{plain}
\begin{center}
%    \Large
%    \textbf{Automated Verification Applied to Stand-alone Solar Photovoltaic Systems}
%    
%    \vspace{0.4cm}
%    \large
%    Optimal Sizing and Project Validation
%    
%    \vspace{0.4cm}
%    \textbf{Alessandro Trindade}
%    
    \vspace{0.9cm}
    \textbf{Abstract}
\end{center}
With declining costs and increasing performance, the deployment of renewable energy systems is growing faster. In 2017, for the first time, the number of people without access to electricity dipped below 1 billion, but trends on energy access likewise fall short of global goals. Particular attention is given to stand-alone solar photovoltaic systems in rural areas or where grid extension is unfeasible. Tools to evaluate or to size electrification projects are available, but they are based on simulations that do not cover all aspects of the design-space. In the other hand, the use of formal methods to model and validate any kind the systems is growing with time, mainly to find bugs in complex hardware and software systems: their aim is to establish system correctness with mathematical rigor. The use of formal methods in electrical systems is a recent subject, with research being published only in the last four years. Moreover, the use of automated synthesis in order to obtain optimal sizing of solar photovoltaic systems was never done before. This Thesis marks two mains achievements: (1) the first application of software model checking to formally verify the design of a stand-alone solar photovoltaic system including solar panel, charge controller, battery, inverter, and electric load; and (2) a sound and automated approach to obtain optimal sizing of stand-alone photovoltaic systems using program synthesis, where every component and function of a solar photovoltaic system is described, including its properties, and the behavioral model that represents the optimal sizing can be synthesized automatically. Related to formal verification, case studies, from real photovoltaic systems deployed in five different sites, ranging from 700 W to 1,200 W, are used to evaluate the proposed approach and to compare that with specialized simulation tool. Different verification tools are evaluated as well, in order to compare performance and soundness. Data from practical applications show the effectiveness of our proposed approach, where specific conditions that lead to failures in a photovoltaic solar system are only detailed by the automated verification method. Moreover, related to the use of program synthesis, we propose a variant of counterexample guided inductive synthesis (CEGIS) approach, with two well-defined phases: first it synthesizes the sizing of stand-alone photovoltaic systems based on power reliability, but that may not achieve the lowest cost; second, the proposed solution is then verified iteratively with a lower bound via symbolic model checking. If the verification step does not fail, the lower bound is adjusted; and if it fails, a counterexample is provided with the optimal sizing, thereby linking the technical response of the first phase with cost analysis of the second phase. Commercial equipment data from different manufacturers are provided to the synthesis engine and candidate solutions are derived from financial analysis of the obtained sizing. The proposed synthesis method is novel and unprecedented to streamline the design of photovoltaic systems. Experimental results using seven case studies show that the proposed synthesis method is able to produce within an acceptable run-time the optimal PV system sizing, and a comparative with a specialized simulation tool and real PV systems shows the effectiveness of approach.



\textit{Keywords}: Formal verification; automated verification; model checking; program synthesis; electrical systems; solar photovoltaic systems
