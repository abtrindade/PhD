\thispagestyle{plain}
\begin{center}
%    \Large
%    \textbf{Automated Verification Applied to Stand-alone Solar Photovoltaic Systems}
%    
%    \vspace{0.4cm}
%    \large
%    Optimal Sizing and Project Validation
%    
%    \vspace{0.4cm}
%    \textbf{Alessandro Trindade}
%    
    \vspace{0.9cm}
    \textbf{Abstract}
\end{center}
With decreasing costs and increasing performance, the deployment of renewable energy systems is now growing faster than in the past decade. In 2017, for the first time, the number of people without access to electricity dipped below 1 billion, but trends in energy access still fall short of global goals. Particular attention is given to stand-alone solar photovoltaic systems in rural areas or where grid extension is unfeasible. Tools to evaluate or to size electrification projects are available, but they are based on simulations that do not cover all aspects of the design space. However, the use of formal methods to model and validate any system has grown with time, mainly to find bugs in sophisticated hardware and software systems: they aim to establish system correctness with mathematical rigor. The use of formal methods in electrical systems is a new subject, with published research spanning only the last four years. Moreover, the use of automated synthesis in order to obtain optimal sizing of solar photovoltaic systems has never been done before. This thesis marks the achievement of two major goals: first, the application of software model checking to verify formally the design of a stand-alone solar photovoltaic system, including solar panel, charge controller, battery, inverter, and electric load; second, a sound, automated approach to obtaining optimal sizing of stand-alone photovoltaic systems using program synthesis. For the formal verification, we used case studies from real photovoltaic systems deployed in five different sites, ranging from $975$ W to $1,300$ W, in order to evaluate the proposed approach and to compare  it with a specialized simulation tool. Different verification tools are evaluated also, in order to compare performance and soundness. Data from practical applications show the effectiveness of our proposed approach, where specific conditions that lead to failures in a photovoltaic solar system are detailed only by the automated verification method. In addition, for the use of program synthesis, we propose a variant of the counterexample guided inductive synthesis (CEGIS) approach. This variant has two phases linking the technical and the cost analysis. First, we synthesize a feasible candidate based on power reliability, but which may not attain the lowest cost. Second, the candidate is then verified iteratively with a lower bound cost via symbolic model checking. If the verification step succeeds, the lower bound is adjusted; if it fails, a counterexample provides the optimal solution. The proposed synthesis method is novel and unprecedented as it streamlines the design of photovoltaic systems. Experimental results using seven case studies demonstrate that our synthesis method can produce optimal system sizing within an acceptable run-time. We also present a comparison with a specialized simulation tool over real photovoltaic systems in order to show the effectiveness of our approach, which can provide a more detailed and accurate solution than the simulation tool.

\textit{Keywords}: Formal verification; automated verification; model checking; program synthesis; electrical systems; solar photovoltaic systems.
