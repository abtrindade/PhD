% This is samplepaper.tex, a sample chapter demonstrating the
% LLNCS macro package for Springer Computer Science proceedings;
% Version 2.20 of 2017/10/04
%
\documentclass[runningheads]{llncs}
%
\usepackage{xcolor}
\usepackage{graphicx}
\usepackage{makecell}
\usepackage{algorithm,algorithmic}
% Used for displaying a sample figure. If possible, figure files should
% be included in EPS format.
%
% If you use the hyperref package, please uncomment the following line
% to display URLs in blue roman font according to Springer's eBook style:
% \renewcommand\UrlFont{\color{blue}\rmfamily}

\renewcommand{\baselinestretch}{0.97}

\begin{document}
%
\title{Automated verification of electrical systems: literature review and case studies of stand-alone solar photovoltaics\thanks{Supported by Newton Fund (ref. 261881580) and FAPEAM
(Amazonas State Foundation for Research Support, calls 009/2017 and PROTI Pesquisa 2018).}}
%
%\titlerunning{Abbreviated paper title}
% If the paper title is too long for the running head, you can set
% an abbreviated paper title here
%
\author{Alessandro Trindade\inst{1}\orcidID{0000-0001-8262-2919} \and
Lucas Cordeiro\inst{2}\orcidID{0000-0002-6235-4272}}
%
\authorrunning{A. Trindade and L. Cordeiro}
% First names are abbreviated in the running head.
% If there are more than two authors, 'et al.' is used.
%
\institute{Federal University of Amazonas, Av. Gen. Rodrigo Octavio, 6200, Coroado I, 69077-000 Manaus AM Brazil \email{alessandrotrindade@ufam.edu.br} \and
The University of Manchester, Kilburn Building, Oxford Road, Manchester M13 UK
\email{lucas.cordeiro@manchester.ac.uk}}
\maketitle              % typeset the header of the contribution
%
\begin{abstract}
Currently around 1 billion of people do not have electrical energy. Energy poverty alleviation has become an important political issue. Several initiatives and policies have been proposed to deal with poor access to sources of energy in many developing countries. Particular attention is given to use renewables and stand-alone solar photovoltaic (PV) systems. Validation tools are available to evaluate PV designs, but they are mainly based on simulations that do not cover all aspects of the design space. The use of automated verification in PV systems is a recent subject with a few initiatives. Here we review state-of-art automated verification techniques for electrical systems and perform their comparative evaluation considering five stand-alone PV systems. We employ a commercial simulator and data extracted from deployed PV systems to validate our research. Experimental results show that our suggested verification engine obtains promising results to detect flaws that other existing approaches are unable to reveal.

\keywords{Automated verification  \and Model Checking \and Electrical systems \and Solar photovoltaic systems.}
\end{abstract}
%

%%%%%%%%%%%%%%%%%%%%%%%%%%%%%%%%%%%%%%%%%%%%%%%%%%%%%%%%
\section{Introduction}
%%%%%%%%%%%%%%%%%%%%%%%%%%%%%%%%%%%%%%%%%%%%%%%%%%%%%%%%

Lack of access to clean and affordable energy is considered a core dimension of poverty~\cite{Hussein2012}. Progress has been made worldwide; in particular, the number of people without electricity access fell below 1 billion threshold for the first time in 2017~\cite{IEAweo2018}. In order to provide universal electricity for all, decentralized systems led by solar photovoltaic (PV) in off-grid and mini-grid systems will be the lowest-cost solution for three-quarters of the additional connections needed; \textcolor{red}{This sentence is unclear... what is the ``role to play''?} however, that grid extension will still have a role to play, especially in urban areas~\cite{IEAweo2018}.

In order to simulate or evaluate a PV system, there are various specialized tools available in the market, e.g. RETScreen, HOMER, PVWatts, SAM and Hybrid2 \cite{Pradhan,Swarnkar,NRELDobos,NRELBlair,Mills}; and even general purpose simulation tools, e.g. PSpice, Saber or MATLAB/Simulink package \cite{Gow1999,Benatiallah2017}. However, those tools are based on simulation models, which are employed before the system design is concluded; it has the drawback of an incomplete coverage since verification of all possible combinations and potential failures of a system is unfeasible~\cite{ClarkeHV18}.

Formal methods based on \textit{model checking} can offer a great potential to obtain a more effective design process of PV systems~\cite{ClarkeHV18}. In this study, we review verification tools for PV systems and also present five case studies, where different model checking tools are compared in terms of performance and reliability when validating stand-alone solar PV systems. In order to compare different model checkers, we developed a mathematical model of each component of a stand-alone PV system, including solar panel, charge controller, batteries, inverter and electrical load. Design requirements as battery autonomy and power demand in addition to weather conditions, as solar irradiance and ambient temperature, are translated as part of that model and automatically checked by the employed model checkers. 

The main contribution of this approach is that it automatically aids to detect design flaws in PV systems by reporting under which conditions a system does not meet the user requirements, thus considerably improving system's reliability~\cite{Akram2018}. Our case studies are empirical; they were deployed in 2018 with feedback information collected from monitoring systems via micro-controllers and by means of interviews with dwellers with the goal of validating our results. 

%%%%%%%%%%%%%%%%%%%%%%%%%%%%%%%%%%%%%%%%%%%%%%%%%%%%%%%%
\section{State-of-art in Automated Verification of Electrical Systems}
%%%%%%%%%%%%%%%%%%%%%%%%%%%%%%%%%%%%%%%%%%%%%%%%%%%%%%%%

The conversion of traditional power grid into a smart grid, a fundamental example of a Cyber-Physical System (CPS), raises a number of issues that requires novel methods and applications. In 2012, a Chinese smart grid implementation was considered as case study to address the verification problem for performance and energy consumption~\cite{Yukseletall2012}. The authors employed a stochastic model checking approach and presented a modelling and analysis study using PRISM~\cite{KwiatkowskaNP11}. The focus of this study was on how CPs integrate information and communication technology functions to the physical elements of a system for monitoring and controlling purposes; here the authors employed automated verification of certain quantitative properties of the system \textcolor{red}{for example?}. %, with no interest in power generation or even solar PV systems.

In 2015, an approach for applying Monte-Carlo simulation to power system protection schemes presented limitations of incomplete coverage of all possible operating conditions~\cite{Sengupta2015}. The authors proposed an automated simulation-based verification technique to verify correctness of protection settings efficiently using hybrid automata-temporal-logic framework. The initial focus was on relay operations and test-case generation to ensure early detection of design errors. %The study was limited to power system protection and did not deal with electricity generation or even solar PV systems.

Other related studies from 2015 include a framework named Modana to achieve an integrated process from modeling with SysML/MARTE to analysis with statistical model checking for CPSs in terms of non-functional properties such as time and energy~\cite{Cheng2015}. In order to demonstrate Modana's capability, the authors modelled energy-aware buildings as a case study, and discussed the analysis on energy consumption in different scenarios. The focus here is on smart buildings and HVAC (heating, ventilation, and air conditioning) systems. %The research did no deal with solar PV systems. 
 
\textcolor{red}{We have to reduce this paragraph:} In 2017, Abate suggested the application of formal methods to verify and control the behavior of computational devices interacting over a shared and smart infrastructure~\cite{Abate2017}. He has shown that formal techniques can provide compelling solutions not only when safety-critical goals are the target, but also to tackle verification and synthesis problems on populations of such devices. This study also argued that alternative solutions based on classical analytical techniques or on approximate computations are prone to errors with global repercussions. Two different areas were identified: thermal loads and electricity-generating devices interacting over a smart energy network or over a local power grid. Abate discussed the aggregation of large populations of thermostatically-controlled loads and of PV panels, and the corresponding problems of energy management in smart buildings, of demand-response on smart grids, and respectively of frequency stabilization and grid robustness. The focus is on controlling the behavior of components, thereby verifying the smart grid as a ``system of systems'' within the context of ``internet of things''. The author also uses approximate model checking of stochastic and hybrid models.

In 2018, a verification methodology was proposed for the Cyber Physical Energy Systems (CPES) with applications to PV panels and its distributed power point tracking~\cite{Driouich2018}. This approach relied on representing unpredictable behavior of the environment to cover all possible feasible scenarios. The simulation results obtained by JModelica covered the system's complete dynamic behavior; however, it was evident the time consuming issue with almost three days of computer effort to verify the design space of one operation hour of the PV panels behavior. %The work did not include the other components of a stand-alone solar PV system.

Another work from 2018 was the approach to modeling smart grid components using a formal specification. The authors used a state-based formal specification language named Z; they demonstrated the application of Z to four smart grid components~\cite{Akram2018}. The presented formal specification can be considered as a first step towards modeling of smart grid using formal methods. The starting point of this study was that a smart home can be considered as an integrated system consisting of various objects and system, which communicates and interacts with each other. This approach is base on Petri nets and from the premise that modeling of the smart home leads to clear understanding of the overall behavior of the smart grid.

%%%%%%%%%%%%%%%%%%%%%%%%%%%%%%%%%%%%%%%%%%%%%%%%%%%%%%%%
\section{Automated Verification Using Model Checking}
\label{sec:AutomatedVerification}
%%%%%%%%%%%%%%%%%%%%%%%%%%%%%%%%%%%%%%%%%%%%%%%%%%%%%%%%

Although simulation and testing explore possible behaviors and scenarios of a given system, they leave open the question of whether unexplored trajectories may contain a flaw. Formal verification conducts an exhaustive exploration of all possible behaviors; when a design is said to be ``correct'' by a formal verification method, it implies that all behaviors have been explored, and questions regarding adequate coverage or missed behavior becomes irrelevant~\cite{Clarke2012}. Formal verification is a systematic approach that applies mathematical reasoning to obtain guarantees about the correctness of a system; one successful method in this domain is model checking~\cite{Clarke2012}. Here we evaluate three state-of-the-art model checkers to formally verifying PV desgins w.r.t. user requirements.

%%%%%%%%%%%%%%%%%%%%%%%%%%%%%%%%%%%%%%%%%%%%%%%%%%%%%%%%
\subsection{CBMC}
%%%%%%%%%%%%%%%%%%%%%%%%%%%%%%%%%%%%%%%%%%%%%%%%%%%%%%%%

The C Bounded Model Checker (CBMC) falsifies assertions in C programs or proves that they are safe if a completeness threshold is given~\cite{Kroening}. CBMC implements a bit-precise translation of a C program, annotated with assertions and with loops unrolled up to a given depth, into a logical formula. If the formula is satisfiable, then a failing execution that leads to a violated assertion exists~\cite{Kroening}. CBMC's verification flow can be summarized in three stages: (i) Front-end: scans, parses and type-checks C code; it converts control flow elements, such as \textit{if} or \textit{switch} statements, loops and jumps, into equivalent guarded \textit{goto} statements, thus aiming to reduce verification effort; (ii) Middle-end: performs symbolic execution by eagerly unwinding loops up to a fixed bound, which can be specified by the user on a per-loop basis or globally, for all loops and finally; (iv) Back-end: supports SAT and SMT solvers to discharge verification conditions.

%%%%%%%%%%%%%%%%%%%%%%%%%%%%%%%%%%%%%%%%%%%%%%%%%%%%%%%%
\subsection{ESBMC}
%%%%%%%%%%%%%%%%%%%%%%%%%%%%%%%%%%%%%%%%%%%%%%%%%%%%%%%%

ESBMC (or Efficient SMT-based Bounded Model Checker) is a bounded and unbounded mdoel checker for C programs~\cite{esbmc2018}, which supports the verification of LTL properties with bounded traces~\cite{DBLP:journals/sosym/MorseCN015}. ESBMC's verification flow can be summarized in three stages: (i) a front-end that can read and compile C code, where the system formal specification is first handled; (ii) preprocessing steps to deal with code representation, control flow and unwinding of loops, and model simplification, thereby aiming to reduce verification effort; and finally (iii) the SMT solving stage, where all constraints and properties of the system are encoded into SMT and checked for satisfiability. ESBMC exploits the standardized input language of SMT solvers (SMT-LIB\footnote{http://smtlib.cs.uiowa.edu/} logic format) to make use of a resource called \textit{assertion stack}~\cite{Morse2015}. This enables ESBMC, and the respective solver, to learn from previous checks, thus optimizing the search procedure and potentially eliminating a large amount of formula state space to be searched, because it solves and disregards data during the process, incrementally. This technique is called ``incremental SMT''~\cite{DBLP:journals/fac/SchrammelKBMTB17} and allows ESBMC to reduce the memory overhead, mainly when the verified system is complex and the computing platform does not have large amount of memory to deal with the entire design space state.

%%%%%%%%%%%%%%%%%%%%%%%%%%%%%%%%%%%%%%%%%%%%%%%%%%%%%%%%
\subsection{Ultimate Automizer}
%%%%%%%%%%%%%%%%%%%%%%%%%%%%%%%%%%%%%%%%%%%%%%%%%%%%%%%%

Ultimate Automizer is an automatic software verification tool for C programs that is able to check safety and liveness properties. The tool implements an automata-based instance of the Counterexample-Guides Abstraction Refinement  (CEGAR) scheme. CEGAR is a technique that iteratively refines an abstract model using counterexamples and use three concepts: precision, feasibility check and refinement~\cite{CEGAR}. In each iteration, Automizer picks a trace (sequence of statements) that leads from the initial location to the error location and check whether the trace is feasible or infeasible. If the trace is feasible, an error is reported to the user; otherwise it is computed a sequence of predicates along the trace as a proof of the trace's infeasibility. The predicates are a sequence of interpolants since each predicate ��``interpolates''�� between a set of reachable states and a set of states from which it cannot reach the error. In the refinement step of the CEGAR loop, Automizer tries to find all traces whose infeasibility can be shown with the given predicates and subtract these traces from the set of (potentially spurious) error traces that have not yet been analyzed. It is used automata to represent sets of traces. The difference to a classical CEGAR-based predicate abstraction is that Automizer has to do any logical reasoning (e.g., SMT solver calls) that involves predicates of different CEGAR iterations~\cite{Automizer2018}. 


%%%%%%%%%%%%%%%%%%%%%%%%%%%%%%%%%%%%%%%%%%%%%%%%%%%%%%%%
\section{Automated verification of Stand-alone Solar PV Systems }
\label{sec:Methodology}
%%%%%%%%%%%%%%%%%%%%%%%%%%%%%%%%%%%%%%%%%%%%%%%%%%%%%%%%

%%%%%%%%%%%%%%%%%%%%%%%%%%%%%%%%%%%%%%%%%%%%%%%%%%%%%%%%
\subsection{Stand-alone Solar PV Systems}
%%%%%%%%%%%%%%%%%%%%%%%%%%%%%%%%%%%%%%%%%%%%%%%%%%%%%%%%


The mathematical modeling of the PV system is based on modular blocks, as illustrated in Fig.\ref{fig:blockdiagram}. %It identifies the PV generator, batteries, charge controller, inverter, and AC load. 
The PV generator, which can be a panel or an array, is a semiconductor device that can convert solar energy into DC electricity. In Fig.\ref{fig:blockdiagram}, there are two variables that depend on the site where the system is deployed and the weather (i.e., solar irradiance $G$ and temperature $T$). For night hours or rainy days, we need to hold batteries where power can be stored and used. The use of battery as a storage form implies the presence of a charge controller~\cite{Hansen}. The PV arrays produce DC and therefore when the PV system contains an AC load, a DC/AC conversion is required. That converter is called of inverter; and the AC load dictates the behavior of AC electrical load from the house that will be fed by the system.
\begin{figure}[h]
\includegraphics[width=0.6\textwidth]{blockdiagramPVS2}
\centering
\caption{Block diagram for a typical stand-alone PV system~\cite{Hansen}.}
\label{fig:blockdiagram} 
\end{figure}
A wide variety of models exists for estimation of power output of PV modules (and $I-V$ or $P-V$ curves) \textcolor{red}{add citations here}. However, this study will rely on the simplified model of 1-diode since prior work previously demonstrated that this model has a small error rate, between 0.03\% and 4.68\% from selected PV panels tested~\cite{Saloux}; the employed model uses formulas from different sources available in the literature~\cite{Hansen,Ross,Saloux}.

In order to model the batteries, it was used the most common one based on lead-acid~\cite{Copetti}. Related to the charge controller, it was used the 4-steps controller processes with MPPT (maximum power point tracking) to decide if the PV array must be disconnected from the system, or if the PV can be reconnected, or if the load is disconnected and when the load is reconnected~\cite{Hansen}. The inverter model is based on efficiency, i.e., the relation between the AC output power and the DC input power of the component~\cite{Hansen}. The AC load, from every case study, is defined by a 24 hour curve, which reflects the estimated use of all electrical loads from the house. Load information was collected during a presential survey performed in each house.

\subsection{Automated Verification Methodology}
In order to evaluate the automated tools when verifying stand-alone solar photovoltaic systems, we created a methodology where each tool will process the ANSI-C code with constraints %($C$) 
and properties %($P$) 
from the PV system that are provided by the user, and the tool will automatically verify if the PV system requirements are met. If it returns a failure (i.e., SAT), then the tool provides a counterexample, i.e., a sequence of states that leads to the property violation. However, if the verification succeeds (i.e., UNSAT), there is no failure up to the bound $k$; therefore, the PV system will present its intended behavior up to the bound $k$.

The methodology can be described as a three stage process. In Step 1, the PV input data (e.g., load power demand and load energy consumption) and the formulae to check the sizing project, the mathematical model, the limits of the weather non-deterministic variables, are all written as an ANSI-C code~\cite{ANSI2018}. In Step 2, the sizing check of the PV system takes place to make sure that the components were selected according to critical month method~\cite{Pinho}. %%%The equations to check the PV sizing were not presented at this paper, however the standard from Engineering Manual was the reference~\cite{Pinho}.
%
In Step 3, weather variables (e.g., solar irradiance and ambient temperature) will be systematically explored by our verification engine based on maximum and minimum values from the site, where the PV system will be deployed. 
%As a consequence, all the formulae of the employed mathematical models will also be updated. 
In addition, depending on one of the desired properties of the system such as battery autonomy, energy availability, or even system power supply, our verification engine is able to indicate a failure if those properties are not met; in this particular case, it provides a diagnostic counterexample that shows in which conditions the property violation occurred. 
%; as the  state of charge of the batteries, load demand of power and the load consumption of energy if defined by the code
% (as reliability, performance, or safety)
%
%\textcolor{red}{In the following paragraph you should related the output of our verification engine with the description of the BMC SAT or UNSAR given above. For example, what does a failure mean? is it SAT?}
%, i.e., our verification engine does not give any guarantee that there is no error in bound $k+1$ unless some induction method is employed~\cite{DBLP:journals/sttt/GadelhaIC17}.
%
%
%---------------------------------------------------------------------
% \subsection{The case studies and the Algorithm}
%---------------------------------------------------------------------
%
% 
%and as backup at night 
%
%The 700 W system: 3 x 325 W PV panels connected in series, controller of 150 V/35 A with a DC-bus of 24 V, 4 x 220 Ah batteries (2 in series and 2 in parallel arrangement), and inverter of 700 W. 
%
%And the 1,200 W PV system: 4 x 325 W connected in series PV panels, with controller of 150 V/35 A  in a DC-bus of 48 V, 4 batteries of 120 Ah connected in series, and a 1,200 W inverter.
%
%As demonstrated at this work, the performance of the system is highly dependent of solar irradiance and temperature, that are specific of the deployed local (latitude and longitude). 

Algorithm~\ref{alg:verification-algorithm} describes the pseudo-code used to perform the automated verification. The batteries are assumed to have initial SOC of 80\% (Line 1). Information related to average temperature ($T$) and solar irradiance ($G$), maximum and minimum annual, are given to the algorithm in Lines 7 to 10 using non-deterministic variables from ESBMC to explore all possible states and the \textit{assume} macro to constrain the non-deterministic values using a given range. In order to reduce the computational effort of the algorithm, every 24h-day was considered as a time-step of 1 hour, and it was split into two parts: (a) one where it is possible to occur PV generation, during daylight, but dependent on the sun and weather conditions; and (b) one that includes all the remaining day (without PV generation). Therefore, our approach depends on specific data about the solar irradiation levels to define the average amount of hours of PV generation.

After that, the model from PV generator is used in the function call of Line 7, to produce the voltage and current considering the states of $G$ and $T$. With respect to every hour considered, the conditional \textit{if-else-endif} statements from Lines 8, 11, 13, 16 and 19, will perform the charge or discharge of batteries according to the value of different variables: if there is PV generation (which depends on $G$ and $T$), the updated state of charge from batteries, the house's load and the set-up information of the PV system.

Next, representing the time of the day where PV generation is not possible anymore, starting in Line 22, the algorithm will only discharge the batteries (using the 1 hour step) until a new charging process (at the next day) starts. Specific \textit{assert} in Line 26 will check the state of charging from batteries, and they will violate the property if their levels reach the minimum that represents a discharged battery; therefore, the PV system is unable to supply energy to the house. Nevertheless, if the verification engine does not fail, then we can conclude that the PV system does not need further corrections up to the given bounds.
%
%\textcolor{red}{this sentence is unclear... After this process is started the battery autonomy verification, from line 31}. \textcolor{red}{this sentence is unclear... Based on the fact that won't be PV generation after a given time of the day, the algorithm will only discharge the batteries until a new charging process (at the next day) to start.} \textcolor{red}{what do you mean by The formal verification is guaranteed?...  The formal verification is guaranteed by  macro to specific variables of the model, according lines 27 and 35.}
% and the non-deterministic variables $G$ and $T$ are considered during the formal verification of the system, otherwise, during the other two periods, there is no PV generation and just the power consumption from the backup batteries. 
%Within this 8h-period, $G$ and $T$ are automated verified with different values every one hour.
%, and change their value every 1 h according with the algorithm created using the technique.
 \begin{algorithm}
 \caption{Model checking algorithm for stand-alone PV}
 \begin{algorithmic}[1]
 \renewcommand{\algorithmicrequire}{\textbf{Input:}}
 \renewcommand{\algorithmicensure}{\textbf{Output:}}
  \STATE $SOC \leftarrow 80\%$ \\
%  \COMMENT {Starting with the PV generation time}
% \\ 
 \textit{LOOP Process}
  \FOR {$h = 1$ to $Hours \, of \, PV \, generation$}
  \STATE $G \leftarrow nondet \textunderscore uint(\,)$ \COMMENT {$G$ is non-deterministic variable}
  \STATE $T \leftarrow nondet \textunderscore uint(\,)$ \COMMENT {$T$ is non-deterministic variable}
  \STATE assume ($Gmin \leq G \leq Gmax$) \COMMENT {restricting $G$ values}
  \STATE assume ($Tmin \leq T \leq Tmax$) \COMMENT {restricting $T$ values}
  \STATE $Imax, Vmax \leftarrow PVgenerationMODEL (G,T)$ 
  \\
%  \COMMENT {If-then-else sequence to imitate charge controller work}
  \IF {($battery \, is \, empty$) AND ($PV \, is \, generating$)}
    \STATE charge battery in 1h
    \STATE PV feed the house
  \ELSIF {($battery \, is \, empty$) AND NOT($PV \, is \, generating$)}
    \STATE FAIL with assert macro
  \ELSIF {NOT($battery \, is \, empty$) AND ($PV \, is \, generating$)}
    \STATE stop battery charge
    \STATE PV feed the house
  \ELSIF {NOT($battery \, is \, empty$) AND NOT($PV \, is \, generating$)}
    \STATE discharge battery in 1h
    \STATE Battery feed the house
  \ENDIF
  \STATE $h \leftarrow (h+1)$
  \ENDFOR
 \\ \textit{Start of battery autonomy verification:}
%%  initialization\;
\STATE $AutonomyCount \leftarrow 1$
 \WHILE {$AutonomyCount \leq autonomy$}
  %%\FOR {$Autonomy = 1$ to $auton$}
  \STATE $SOC \leftarrow SOC - ( 24 - Hours \, of \, PV \, generation)\,h\, discharge$
  \STATE $AutonomyCount \leftarrow AutonomyCount+1)$
  \\  
%    \COMMENT {autonomy verification during discharge period}
%  \\
  \STATE assert NOT($Battery \, empty$)  
  \\
  \COMMENT {Perform similar \textbf{for}-LOOP as defined in line 2-21}
  %%\ENDFOR
  \ENDWHILE
 \RETURN $(\,)$ 
 \end{algorithmic} 
 \label{alg:verification-algorithm}
 \end{algorithm}

%%%%%%%%%%%%%%%%%%%%%%%%%%%%%%%%%%%%%%%%%%%%%%%%%%%%%%%%
\section{Experimental Evaluation} 
\label{sec:results}
%%%%%%%%%%%%%%%%%%%%%%%%%%%%%%%%%%%%%%%%%%%%%%%%%%%%%%%%

%%%%%%%%%%%%%%%%%%%%%%%%%%%%%%%%%%%%%%%%%%%%%%%%%%%%%%%%
\subsection{Description of the Case Studies}
%%%%%%%%%%%%%%%%%%%%%%%%%%%%%%%%%%%%%%%%%%%%%%%%%%%%%%%%

We have performed five case studies to evaluate the tools, as described by Table~\ref{tab2}. 
\begin{table}
\caption{Case studies: stand-alone solar PV systems.}\label{tab2}
\begin{tabular}{|c|c|c|c|c|c|}
\hline
\hline
Item & House 1 & House 2 & House 3 & House 4 & House 5\\
\hline
\hline
PV Panels &  \multicolumn{4}{|c|}{3x 325W: in series} & 4x 325W: 2x series and 2x parallel \\
\hline
Batteries & \multicolumn{4}{|c|}{\makecell{4x220Ah: 2x series and 2x parallel\\ autonomy: 48h}} & \makecell{4x 120Ah batteries in series\\ autonomy: 6h}\\
\hline
Charge Controller & \multicolumn{5}{|c|}{With MPPT of 150V/35A}\\
\hline
Inverter & \multicolumn{4}{|c|}{700W with surge of 1,600W} & 1,200W with surge of 1,600W\\
\hline
Power demand & 253W & 263W & 283W & 501W & 915W\\
\hline
GPS Coordinates & \multicolumn{4}{|c|}{\makecell{2$^{o}$44'50.0"S 60$^{o}$25'47.8"W\\}} & \makecell{3$^{o}$4'20.208"S \\60$^{o}$0'30.168"W}\\
\hline
Details & \multicolumn{4}{|c|}{\makecell{Riverside indigenous community\\Rural Area of Manaus - Brazil}} & \makecell{Urban house \\Manaus-Amazonas-Brazil}\\
\hline
\hline
\end{tabular}
\end{table}

The annual average temperature ($T$) in Manaus is from 23$^{o}$C to 32$^{o}$C; and irradiance ($G$) varies from 274 W/m$^{2}$ to 852 W/m$^{2}$ when there is sunlight (Lines 5 and 6 of Algorithm~\ref{alg:verification-algorithm}). Another characteristic of Manaus, based on historical weather data %\cite{Temperature}, 
\cite{Irradiance}, is related to the fact that only during 8 hours of the day is possible to have PV generation, from 8:00h to 16:00h. % (that information is provided in Algorithm~\ref{alg:verification-algorithm} as well).

%%%%%%%%%%%%%%%%%%%%%%%%%%%%%%%%%%%%%%%%%%%%%%%%%%%%%%%%
\subsection{Objectives and Setup}
%%%%%%%%%%%%%%%%%%%%%%%%%%%%%%%%%%%%%%%%%%%%%%%%%%%%%%%%

Our experimental evaluation aims to answer two research questions:
%
\begin{enumerate}
\item[RQ1] \textbf{(soundness)} Does the automated verification provide correct results?
\item[RQ2] \textbf{(performance)} How does the  tools compare each other?
\end{enumerate}

It was used two different setups of hardware and different verification engines, combined with different Solvers. The aim was to evaluate different environments and possible user limitation in terms of main processor and RAM memory. All the experiments were performed without a predefined timeout.

Mostly of experiments were conducted on an otherwise idle Intel octa core Xeon CPU E5-2640 with 2.4 GHz and 264 GB of RAM, running Ubuntu 18.04.1 LTS 64-bits. It was called high-end setup and the verification engines demanded until 90G bytes of RAM.

In addition, it was considered a low-end setup, using specifically the incremental option of ESBMC verification engine, where the demand of memory is low (2G Bytes of RAM). The experiments were conducted on an Intel Core i7-2600 (8-cores), with 3.4 GHz and 32 GB of RAM, running Ubuntu 18.04.1 LTS 64-bits. 

Concerning the high-end configuration: (i) verification engine ESBMC, version v5.1.0 was used with the SMT solver Boolector version 2.4.1~\cite{Brummayer}\footnote{\$ esbmc filename.c -\phantom{}-no-bounds-check -\phantom{}-no-pointer-check -\phantom{}-no-div-by-zero-check -\phantom{}-unwind 300 -\phantom{}-boolector}; (ii) verification engine CBMC 5.6 and MiniSat~\cite{Kroening}\footnote{\$ cbmc filename.c --unwind 300}; (iii) Ultimate Automizer version 0.1.24~\cite{Automizer2018}\footnote{\$ ./Ultimate -tc config/AutomizerReach.xml -i filename.c}.
 
% enabled with the goal of reducing memory usage; 
%The experiments were performed without a predefined timeout.
Concerning the low-end configuration: ESBMC v5.1 was used with the SMT incremental mode\footnote{\$ esbmc filename.c -\phantom{}-no-bounds-check -\phantom{}-no-pointer-check -\phantom{}-no-div-by-zero-check -\phantom{}-unwind 300 -\phantom{}-smt-during-symex -\phantom{}-smt-symex-guard -\phantom{}-z3} enabled with the goal of reducing memory usage; we have also used the SMT solver Z3 version 4.7.1~\cite{DeMoura}. 

In order to evaluate the verification methods and its performance, we have considered five case studies and also compared its performance to the HOMER Pro tool. %Every dweller, who owns a PV system, was interviewed to get information about his/her PV system during four months of use. This information was used to know possible flaws from every system in the field.
Experimental setup of HOMER Pro: Intel Core i5-4210 (4-cores), with 1.7 GHz and 4 GB of RAM, running Microsoft Windows 10; we have used HOMER Pro v3.12.0. 

%\textcolor{red}{What are the command-lines used for all tools? Versions? Solvers? Solvers versions?}

%%%%%%%%%%%%%%%%%%%%%%%%%%%%%%%%%%%%%%%%%%%%%%%%%%%%%%%%
\subsection{Results}
\label{sec:results_indeed}
%%%%%%%%%%%%%%%%%%%%%%%%%%%%%%%%%%%%%%%%%%%%%%%%%%%%%%%%

%
Table~\ref{tab1} summarize the results concerning the use of the automated verification tools applied to the five case-studies of a solar PV system. The time expressed at the Table answer the RQ2. Worth to mention that using simulation tool (HOMER Pro) in all case studies, the 1,200W was the only one that was proved to not meet the requirement of battery autonomy; all the 700W systems had no indication of flaws during simulation. The simulation took less than five seconds to be performed in each case study.

\begin{table}
\centering
\caption{Summary of the case-studies comparative and the automated tools.}\label{tab1}
\begin{tabular}{|c|c|c|c|c|}
\hline
\hline
\multicolumn{5}{|c|}{Model Checker (SAT/UNSAT: time and message)}\\
\hline
Case &  \makecell{ESBMC \\(Boolector)} & \makecell{CBMC\\(MiniSat)} & UAutomizer & \makecell{ESBMC Incremental\\(Z3)}\\
\hline
\hline
House 1 &  \makecell{7.44h \\(low SOC after 16:00h)} & \makecell{Inconclusive \\(during solver)} & XX & \makecell{409.3h \\ (low SOC after 16:00h)}\\
\hline
House 2 &  \makecell{5.74h \\(low SOC after 16:00h)} & \makecell{Inconclusive \\(during solver)} & XX & \makecell{611.2h \\ (low SOC after 16:00h)}\\
\hline
House 3 & \makecell{Inconclusive \\(during solver)} & \makecell{Inconclusive \\(during solver)} & XX & \makecell{615.8h \\ (low SOC after 16:00h)}\\
\hline
House 4 & \makecell{5.52h \\(low SOC after 16:00h)} & \makecell{Inconclusive \\(during solver)} & XX & \makecell{620.8h \\ (low SOC after 16:00h)}\\
\hline
House 5 & \makecell{0.55h \\(Sizing: number of panels)} & \makecell{Inconclusive \\(during solver)} & XX & \makecell{63.3h \\ (Sizing: number of panels)}\\
\hline
\hline
\end{tabular}
\end{table}

The engines ESBMC and ESBMC Incremental accused that the 1,200W system (House 5) has sizing error related to the right number of solar panels. Related to the 700W systems (Houses 1-4) the verification engines ESBMC and ESBMC incremental indicate that there is a situation of low state of charge from batteries that will cause the cut of the load, just after the 16:00h of the day, when the weather condition was not enough to charge the batteries.

The CBMC tool, was not successful to produce results. It finished the verification of all cases, but the log file indicated that the Solver did not present SAT or UNSAT answer. We tried the MiniSAT and even the Boolector Solvers.

Related to UAutomizer, ....

The flaws from 700W systems were confirmed with the dwellers who own the systems by an interview process and from a local monitoring system with the Raspberry Pi controller: at least once a month is usual the system to turn off, normally in raining or clouds days, further affirming RQ1; related to the 1,200W system, we identified that, in fact, it does not meet the battery autonomy most of the time (mainly when all loads are turned on), and this was double checked with the monitoring system from the charge controller from the manufacturer who showed that the maximum power or surge power were not exceeded, thus affirming RQ1; this behavior is expected since the system was purchased as an off-the-shelf solution and not as a customized design for the electrical charges of the house

%%%%%%%%%%%%%%%%%%%%%%%%%%%%%%%%%%%%%%%%%%%%%%%%%%%%%%%%
\section{Conclusions and Future Work}
%%%%%%%%%%%%%%%%%%%%%%%%%%%%%%%%%%%%%%%%%%%%%%%%%%%%%%%%

We have described and evaluated different automated verification methods to check whether a given PV system meets its specification using software model checking techniques. We have considered five case studies from real photovoltaic systems deployed in five different sites, ranging from $700$ W to $1,200$ W. ESBMC had the better results, in both setups (high-end and low-end configurations). Although the verification method takes longer than simulation methods, it is able to find specific conditions that lead to failures in a PV system previously validated by a commercial simulation tool. In particular, the proposed method was successful in finding sizing errors and indicating in which conditions a PV system can fail. Future work will focus in other verification engines and different solvers, besides the improvement of the algorithm in order to obtain better performance.
%
% ---- Bibliography ----
%
% BibTeX users should specify bibliography style 'splncs04'.
% References will then be sorted and formatted in the correct style.
%
\bibliographystyle{splncs04}
\bibliography{trindadeThesis}
%

\end{document}

%\subsection{Design and Simulation of Solar PV systems}
%The design and validation of a PV system can be done by hand or with the aid of a software tool. In order to address different aspects of the PV system design, there are various software tools available in the literature~\cite{Rajanna,Rawat}.
%public domain and commercial software available for the PV market. 
%According to \cite{Brooks}, t
%%%%%%%%%%
%
%The capabilities of tools available in the literature range from simple solar resource and %energy production estimation, %to site survey and system design tools,
%to complex financial analysis and project optimization. At this study, the commercial simulation tool HOMER PRO was selected to be used at the case studies in order to be compared with the automated verification tools.
