\documentclass[11pt]{article}

\usepackage[normalem]{ulem}
\usepackage[english]{babel}
\usepackage[latin1]{inputenc}
\usepackage{graphicx}
\usepackage{amssymb}
\usepackage{epstopdf}
\usepackage{url}
%\usepackage{mathtools}
\usepackage{enumerate}
\usepackage[nodayofweek]{datetime}
\usepackage[caption=false]{subfig}
%\mathtoolsset{showonlyrefs}
\usepackage[linesnumbered,lined,ruled,boxed]{algorithm2e}
\SetKw{KwBy}{by}
\usepackage{xcolor}
%\input{rgbcolors}
%\usepackage{color}
\usepackage{natbib}
% \usepackage[style=authoryear-comp,dashed=false]{biblatex}
% \bibliography{piecewise_lyapunov}
\newtheorem{myassumption}{Assumption}
\newcommand{\mstitle}{Automated Formal Verification of Stand-alone Solar Photovoltaic Systems}
\newcommand{\refnumber}{SE-D-19-01248R1}
\newenvironment{resposta}{~~~\begin{quote}\color{blue}\textbf{Response:}}{\end{quote}}
\newcommand{\comment}[1]{}
%\definecolor{DarkGreen}{rgb}{0.2, 0.4, 0.2}
\usepackage{amsmath}
\usepackage{listings}
\lstset{language=C,basicstyle=\small\ttfamily}
\usepackage{tikz}
\usetikzlibrary{positioning, automata, shapes.arrows, calc, shapes, arrows, calc,patterns,decorations.pathmorphing,decorations.markings}

\newcommand{\param}[2]{\ensuremath{\langle{#1},{#2}\rangle}\xspace}
\textwidth 15cm
\setlength{\textheight}{1.1\textheight}
\newcommand\hi{\hspace*{\parindent}}
\newcommand\vi{\vspace{\baselineskip}}
\newcommand\lac{{\mbox{{\Huge\bf L}\hspace{-0.65em}
\raisebox{-1.2ex}{\Huge\bf A}\hspace{-1.1em}
\raisebox{-0.6ex}{\Huge\bf C~}}}}
\newcommand{\fwlfunction}[1]{\mathcal{FWL}[#1]}
\usepackage{fancyhdr}
\pagestyle{fancy}
\fancyhf{}
\lhead{Manuscript Reference Number: \refnumber}
\rhead{Page \thepage\ of \pageref{LastPage}}
\lfoot{}
\rfoot{}

%\pagestyle{headings}
\pagenumbering{roman}
\begin{document}
\setcounter{page}{1}
\thispagestyle{empty}


\hoffset -1.5cm \voffset .5cm


\sf

\vspace*{-2.5cm} \hspace*{-0.1cm} {\small
{\mbox{\begin{minipage}{2cm}
\centerline{\includegraphics[width=1.5cm]{ufam_logo.png}}
\end{minipage}}} \hspace*{0.1cm}
\begin{minipage}{11.0cm}{\large \textbf{Federal University of Amazonas}}\\
{\sc Faculty of Technology}\\
{\sc Department of Electricity}
\end{minipage}
} \vspace*{0mm}

\hspace*{-.7cm} {\rule[-1ex]{15cm}{0.03cm}}

\begin{flushright}
\begin{minipage}{7.0cm}\small
{\bf Please Reply to:}\\
Assistant Professor Alessandro Trindade\\
Universidade Federal do Amazonas\\
Departamento de Eletricidade\\
Av. Gal. Rodrigo Ot�vio, 3000, Japiim, Campus Universit�rio\\
ZIP 69077-000, Manaus - AM, Brazil\\
{\em alessandrotrindade@ufam.edu.br}
\end{minipage}
\end{flushright}
 
\vi
\hspace*{\fill}{\small Manaus, \today.}
\vi



\begin{flushleft}
    Mario A. Medina, PhD, PE\\
    Associate Editor\\
	Solar Energy
    \end{flushleft}
\vi

\begin{flushleft}
\textbf{REF.:} \refnumber
    \end{flushleft}
\vi \vi
 
\indent Dear Mario,
\vi 

We thank you for asking us to comply with your reviewers' reports. 
We are submitting a revised version of our highlights file from the entitled paper  
``{\em \mstitle}'' by Alessandro Trindade and Lucas Cordeiro. 
This revised version carefully addresses the comment provided 
by the reviewers, as suggested in your original decision letter. 
In particular, our reply letter describes all modifications we made 
to our highlights file and the respective response to the comments 
raised by the reviewer.  

%\vi

%Thank you very much in advance.

\vi\vi

\indent
Sincerely,\\



\begin{quote}
\begin{quote}
\begin{flushright}


\vi
\vi

Alessandro B. Trindade~~~~~~~
\end{flushright}
\end{quote}
\end{quote}

\hoffset -1.5cm \voffset .5cm


% =============================================
\newpage
\subsection*{Authors'  Response to the Review Comments on manuscript ``{\mstitle}'' -- Manuscript Reference Number: \refnumber}

\vi

The authors would like to thank the anonymous reviewers for their valuable and constructive comments and suggestions related to the highlights file, which helped us significantly improve the quality of our material. 

According to the associate editor and reviewers' comments, our highlights file has been carefully modified and rewritten. 

Our responses  to all comments (in blue color) are given in the sequel, with clear indications about how and where they were addressed.

We hope those modifications in our highlights file and also our responses are sufficient to make our work suitable for publication in {\bf Solar Energy}.



\newpage
\subsection*{Reviewers' comments:}

\begin{quote}
Mandatory Editorial Changes: your highlights are incomplete sentences.

Please follow the Guide for Authors to construct the Highlights.
Your Highlights do not comply with the requirements stated in the Guide for Authors.  Failure to comply with this requirement will result in the rejection of your paper. 

1.     Highlights must be complete sentences with a maximum of 85 characters, not words, including spaces, per highlight.
A complete sentence is a sentence that conveys and idea, contains a subject and a verb predicate, projects a full thought, and is able to stand alone.
Please visit www.elsevier.com/authors/journal-authors/highlights for more details and to see examples of highlights.

To understand the concept of a complete sentences, please visit: 

http://faculty.washington.edu/ezent/imsc.htm
\end{quote}


\begin{resposta} 
We thank the reviewers for this insightful comment and suggestion. Indeed, we had to rewrite the highlights doc file to comply with the correct expectation from {\bf Solar Energy} journal. 

Now our new highlights files states:
\begin{itemize}
\item Verification can be effective to validate stand-alone solar photovoltaic systems
\item Automated verification method can detect design errors better than simulation
\item ESBMC can produce better performance and soundness among automated verifiers
\end{itemize}


Compared with the original and not suitable:
\begin{itemize}
\item Approach detect design errors which are not easily detected by other approaches
\item Algorithm that implements the automated verification method 
\item Formal checking of the sizing and the operation of a given stand-alone PV system
\item Comparative of three state-of-the-art model checkers in five real case studies
\end{itemize}


\end{resposta}



\label{LastPage}

% \printbibliography
\end{document} 
